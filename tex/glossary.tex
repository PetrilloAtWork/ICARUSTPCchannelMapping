\section{Glossary}
\label{sec:glossary}

\begin{description}
  \item[Collection plane] is the wire plane farthest from the cathode,
    with most of the wires read on top of the anode frame and a few read on one
    of the anode frame sides.
  \item[Cryostat] in \LArSoft terms, each of the liquid argon volumes, \ie the
    two modules.
  \item[First induction plane] is the wire plane closest to the cathode,
    with 9-meter long wires being read out on the sides of the anode frame.
  \item[Flange] one of the 96 interfaces between cold and warm volumes
    \emph{serving the TPC wire planes}
    (other flanges, \eg the ones serving the photomultipliers, are never
    considered in this document).
  \item[Flat cable] each of the 68-wire twisted-pair cables connecting one side
    of the decoupling and biasing board to a group of 32 contiguous TPC wires
    at the anode frame.
  \item[Minicrate] metal crate hosting nine readout boards,
    and installed on the warm side of each of the 96 flanges.
  \item[Module] each of the two liquid argon tanks (``T300''), identified by
    a relative geographic label \emph{East} and \emph{West}; sometimes it is
    called \emph{cryostat}.
  \item[Second induction plane] is the middle wire plane,
    with most of the wires read on top of the anode frame and a few read on one
    of the anode frame sides.
\end{description}
