\section{Glossary}
\label{sec:glossary}

\begin{description}

  \item[Anode frame]
    the part of the TPC providing support to the three wire
    planes. It is the interface between the TPC wires and the flat cables.

  \item[Board]
    usually refers to a \emph{readout board}.

  \item[Cable]
    usually refers to a 68-wire \emph{flat cable}.

  \item[Cathode]
    the part of the module set at high voltage ($75\,\textnormal{kV}$)
    and shared by two TPC.

  \item[Channel]
    a gateway to confusion. It refers to a \emph{readout channel},
    which can be identified according to two completely different systems:
    a \hyperlink{glossary:LArSoftChannelID}{\emph{LArSoft channel identifier}},
    which is unique in the detector (\texttt{0}-\texttt{55295}),
    and a \hyperlink{glossary:ReadoutChannelNumber}{\emph{readout channel number}}
    within a readout board,
    which is unique only within its readout board (\texttt{0}-\texttt{63}).
  
  \item[Chimney]
    the tube steel structure hosting the flanges. There are 20 chimneys assigned
    to each TPC, and they are labeled with the module and TPC label plus a
    sequence number from 1 to 20, e.g. \Chimney{EW04}. Chimneys number 1
    (\Chimney{EE01}, \Chimney{EW01} etc.) are located in the south side of the
    detector. Chimneys are categorized as ``standard'' (or ``short'')
    ``end'' (or ``non-standard'', ``corner'', or ``tall'').

  \item[Collection plane]
    is the wire plane farthest from the cathode,
    with most of the wires read on top of the anode frame and a few read on one
    of the anode frame sides. It has the highest bias voltage.

  \item[Corner chimney]
    the same as \emph{end chimney} (they are at the corners of a single module).

  \item[Cross]
    a synonym for \emph{chimney}: standard chimneys present one TPC flange at
    the top and two smaller opposite flanges on the shaft, making chimney
    profile cross shaped.

  \item[Cryostat]
    in \LArSoft terms, each of the liquid argon volumes, \ie the
    two modules.

  \item[DBB] (\emph{Decoupling and Biasing Board})
    a passive board attached
    to the cold side of a flange, distributing the bias voltage to the wires
    and conveying their signal to the readout boards. Each board can serve
    64 TPC wires connected via two flat cables.

  \item[End chimney]
    any of the eight chimneys at the north and south borders of the detector:
    \Chimney{EE01}, \Chimney{EW01}, \Chimney{WE01}, \Chimney{WW01} (all south side),
    \Chimney{EE20}, \Chimney{EW20}, \Chimney{WE20} and \Chimney{WW20} (all north side).
    They convey information collected from the side of the anode frame, that is
    from all the wires on half of the first induction wire plane,
    plus the short wires of one of the other wire planes.
    To accommodate this, they include three flanges, the top and middle ones
    serving the first induction plane and the bottom one serving the other plane.
    These chimneys may be also called non-standard, corner or tall chimneys.

  \item[First induction plane]
    the wire plane closest to the cathode,
    with 9-meter long wires being read out on the sides of the anode frame.
    It has the lowest bias voltage among the wire planes.

  \item[Flange]
    one of the 96 interfaces between cold and warm volumes
    \emph{serving the TPC wire planes}
    (other flanges, \eg the ones serving the photomultipliers, are never
    considered in this document).
    Flanges are identified by their chimney label (\eg \Flange{EW04}), with the
    addition of an additional tag for the ones on the end chimneys,
    to denote the bottom flange (\texttt{EW01b}),
    the middle one (\texttt{EW01m}) or the top one (\texttt{EW01t}).

  \item[Flat cable]
    each of the 68-wire twisted-pair cables connecting one side
    of the decoupling and biasing board to a group of 32 contiguous TPC wires
    at the anode frame, each via a single twisted pair (two additional pairs are
    not connected to any TPC wire). The most external of the 34 pairs connected
    to a TPC wire is tagged as pair \#1, its wire is wrapped in red, and its
    position is marked on the connector. The connectors present an array of
    $34 \times 2$ pins, and each pair is connected to a $1 \times 2$ pair of
    pins.

  \item[Ghost channel]
    a readout channel that is not connected to a flat cable.
    This happens in a few readout boards and DBB on the end chimneys:
    one board for each bottom flange (connected to only 17 cables)
    and one board for each the top flange of the end chimney which serve the
    first induction plane, whose eighth slot is connected to a flat cable only
    by one of its two sides.
  
  \item[\hypertarget{glossary:LArSoftChannelID}{LArSoft channel identifier}]
    (also referred to as \emph{channel} or \emph{channel ID}, both ambiguous)
    identifies a TPC readout channel uniquely in the whole detector
    (\texttt{0}-\texttt{55295}).
  
  \item[\hypertarget{glossary:minicrate}{Minicrate}]
    metal crate hosting nine readout boards,
    and installed on the warm side of each of the 96 flanges.
    A few flanges are connected to 16 cables or less, and therefore have
    less than nine operational readout boards.

  \item[Module]
    each of the two liquid argon tanks (``T300''), identified by
    a relative geographic label \emph{East} (\Module{E}) and \emph{West}
    (\Module{W}); sometimes it is called \emph{cryostat}.

  \item[Non-standard chimney]
    another name for \emph{end chimney}.
    Out of 80 chimneys, 72 are short, which makes the remaining ones ``non-standard''.

  \item[Readout board]
    each of the boards hosting the TPC channel readout, including the ADC.
    Each board is connected through a DBB to two flat cables, serving 64 channels.
    Boards are numbered from \Board{0} to \Board{8}, with board \Board{8}
    the one closest to the triangular mark on the flange
    (\cref{fig:FlangeConnectionsCorner}).

  \item[\hypertarget{glossary:ReadoutChannelNumber}{Readout board channel number}]
    (also referred to as \emph{channel} or \emph{channel number}, both ambiguous)
    identifies a TPC readout channel within its own readout board
    (\texttt{0}-\texttt{63}).
  
  \item[Readout channel]
    a channel in the readout board. All channels are connected to a DBB.
    The channels not connected to a flat cable are \emph{ghost channels}.
    The channels connected to a flat cable but not to a TPC wire are
    \emph{wireless channels}. The definition of channel is stretched to include
    non-entities, the \emph{virtual channels}, from minicrate slots that might
    host readout boards but are actually empty.
  
  \item[Second induction plane]
    is the middle wire plane,
    with most of the wires read on top of the anode frame and a few read on one
    of the anode frame sides.
    Its bias voltage is also intermediate between the other two planes.

  \item[Short chimney]
    another name for a \emph{standard chimney}. These chimneys are shorter
    than the end ones because they host only one TPC flange each.

  \item[Short wire]
    any of the TPC wires on second induction and collection planes which are
    fixed to one of the sides of the anode frame and as a consequence are
    shorter than the wires fixed at top and bottom sides on the same plane
    (which are about \meter{3.7} long).

  \item[Slot]
    the location of a readout board in a minicrate. Each crate has nine slots.
    In previous tests, slots used to be numbered from 1 to 9, which now host
    boards from 0 to 8 respectively.

  \item[Standard chimney]
    any of the chimneys hosting a single TPC flange (together with other flanges
    for PMT, which are not relevant here). These flanges are all connected to
    cables at the top of the anode frame, serving the second induction or the
    collection planes. There are 72 of these chimneys,
    numbered from 2 to 19 from south to north (\eg \Chimney{EE19} is the
    standard chimney of TPC \TPC{EE} closest to the north side).

  \item[Tall chimney]
    another name for a \emph{end chimney}. These chimneys are taller than the
    standard ones because they stack three TPC flanges each instead of just one.

  \item{TPC}
    (Time Projection Chamber) the volume delimited by a cathode and an
    anode; there are two TPC in each module, and they share the cathode.
    TPC are geographically identified by their position relative to the cathode,
    as \emph{East} (``E'') and \emph{West} (``W''); to uniquely identify a TPC,
    a label including also its module is used: TPC \TPC{EW} is the west TPC
    belonging to the east module. Therefore the four TPC are labeled, from west
    to east, \TPC{WW}, \TPC{WE}, \TPC{EW} and \TPC{EE}.

  \item[TPC wire]
    (sometimes just \emph{wire}) each of the conducting wires
    crossing the anode and set at a certain bias voltage.
    Each of the wires is connected to a readout channel via a flat cable and DBB.

  \item[Twisted pair]
    a pair of contiguous wires in a flat cable, twisted around
    each other. Thirty-two of the pairs have one wire connected to a TPC wire,
    and the other grounded at the DBB end providing some electric shielding.

  \item[Virtual channel]
    a channel that would arise from a readout board in a slot that is instead
    empty. The top flange of each end chimney has an empty slot without readout
    board, and therefore it has 64 virtual channels.
  
  \item[Wireless channel]
    a non-ghost channel that is not connected to a TPC wire.
    This happens at the corners of the second induction and collection planes
    where the wires are short (bottom flanges in all the end chimneys, and all
    flanges on chimneys number 2 and 19), where the 64 shorter wires are
    actually not physically present.
  
  \item[Wire plane]
    (or just \emph{plane}) a set of TPC wires all with the same
    orientation, lying on the same geometrical plane and sharing the same
    electrical potential.
    Depending on their position (and bias voltage), they are labeled first
    induction, second induction and collection planes.

  \item[Wire]
    usually same as \emph{TPC wire}); according to the context,
    it may also refer to one of the conductors of a flat cable.

\end{description}
