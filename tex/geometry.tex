\section{Detector geometry}
\label{sec:geometry}

In this section we describe the relevant components of the detector and their relations.

The detector is made of two identical modules
identified by their relative geographic location
as \emph{East} (\Module{E}) and \emph{West} (\Module{W}) module (\cref{fig:chimneys}).
As a reminder, the orientation is such that the Booster Neutrino Beam impinges
upon the detector from the south side.

\begin{figure}[bt] % [p]
%   \centering{\includegraphics[width=0.90\textheight,angle=270]{figures/Icarus_chimneys}}
  \includegraphics[width=\textwidth]{figures/ChimneyMap}
  \caption{
    Disposition of the chimneys on the top of ICARUS detectors (from~\cite{SBNDocDB14316}; see also~\cite{SBNDocDBxxxx:ConnTest}).
    \label{fig:chimneys}
  }
\end{figure}

In the following text we'll describe module \Module{E},
with the understanding that the same considerations apply to module \Module{W} as well.


\subsection{TPC wires}
\label{ssec:geometry:wires}

The module internally presents the vertical cathode in the middle, shared by
the two TPC, \TPC{EW} and \TPC{EE}.
The frame of the anodes pin the TPC wires on three different planes.
As shown in \cref{fig:WirePlanesInTPC}, the most internal of the planes,
the \emph{first induction plane}, has horizontal wires crossing the anode frame.
These wires are actually about \meter{9} long, with one end fixed at the middle
of the frame and the other end fixed to one of the vertical short sides of the anode frame.
We can label them as ``north wires'' or south wires
according to the side of the anode frame they are connected to.
These wires are connected via 68-pin cables to the top and middle flange
of the closest end chimney.
The cables have tag
\CableTag{C} (\texttt{EE01} and \texttt{WE01}),
\CableTag{D} (\texttt{EE20} and \texttt{WE20}),
\CableTag{A} (\texttt{EW01} and \texttt{WW01})
and
\CableTag{B} (\texttt{EW20} and \texttt{WW20}).

The other two planes have most of the wires fixed at the top and the bottom of the anode frame
(\centim{365.8} long).
These wires are connected to the 68-pin cables on the top side of the anode frame.
Other wires are fixed at the top and at a side of the frame:
these shorter wires are also connected to the cables on the top side of the anode frame.
The cables from both these groups of wires lead to the single flange of the closest of the standard chimneys.
Finally, the rest of the wires are fixed at the side and the bottom of the anode frame.
The cable of these wires lead to the bottom flange of the closest end chimney.


\subsection{Flanges}

The flanges on the standard chimney each serve nine cables from the second induction plane
and nine cables from the collection plane.
The connectors are deployed in a $9 \times 2$ array (\cref{fig:FlangeConnectionsStandard}).

\begin{figure}
  \begin{subfigure}{0.45\linewidth}
    \includegraphics[width=\textwidth]{figures/ReadoutCrate-standard}
    \label{fig:StandardMinicrate}
  \end{subfigure}
  \begin{subfigure}{0.55\linewidth}
    \includegraphics[width=\textwidth]{figures/TopFlangesAndMinicrate}
    \label{fig:FlangeConnectionsStandard}
  \end{subfigure}
  \caption{
    Picture and illustration of the connections on top of a flange on a standard chimney.
    The orientation is set by the reference mark,
    the yellow triangle on the top right,
    which is pointing toward the side of the cathode
    (\ie westward for \TPC{E} TPCs and eastward for \TPC{W} TPCs).
  }
\end{figure}

These cables may come with a \CableTag{V} or \CableTag{S} tag, depending on the TPC.
Their cables are numbered from 1 to 18 (\eg \Cable{S13}):
cables from 1 to 9 belong to the west-most plane, while the other belong to the east-most one.

The eight end chimneys are more varied, each one with three flanges.
The top two flanges (``top'' and ``middle'') serve the first induction plane,
while each of the bottom ones serves one of the other two planes.

The minicrate on these flanges is indistinguishable from the ones on the standard chimneys,
with the connectors still deployed in a $9 \times 2$ array
(\cref{fig:FlangeConnectionsCorner})
but with different connection patterns.

\begin{figure}
  \begin{subfigure}{0.50\linewidth}
    \includegraphics[width=\textwidth]{figures/CornerFlangesAndMinicrate_upward}
    \caption{Actual cable tags: \CableTag{A} (\Chimney{xW20}), \CableTag{C} (\Chimney{xE01}).}
    \label{fig:FlangeConnectionsCorner_upward}
  \end{subfigure}
  \begin{subfigure}{0.50\linewidth}
    \includegraphics[width=\textwidth]{figures/CornerFlangesAndMinicrate_downward}
    \caption{Actual cable tags: \CableTag{B} (\Chimney{xW01}), \CableTag{D} (\Chimney{xE20}).}
    \label{fig:FlangeConnectionsCorner_downward}
  \end{subfigure}
  \caption{
    Illustration of the connections on top of a flange on a end chimney.
    The orientation is set by the reference mark.
    To the standing observer, half of the minicrates looks upside down.
    The yellow triangle still points toward the side of the cathode
    (\ie westward for \TPC{E} TPCs and eastward for \TPC{W} TPCs).
  }
  \label{fig:FlangeConnectionsCorner}
\end{figure}

The labelling looks different for end chimneys at different locations.
To better understand their orientation,
it is important to consider that the physical minicrate has one side
(with fans and power supply connection) which is always oriented toward the border of its module (\ie toward the anode).
Since on the two ends of the TPC the flanges are rotated by the right angle around the east-west direction ($x$ direction),
one clockwise and the other counterclockwise,
the effect is that to the standing observer they look one upside down respect to the other.
Removing this effect, though, they are actually placed in the same way.



\begin{figure}[p]
  \centerline{\includegraphics[width=0.90\textheight,angle=90]{figures/Icarus_TPC_wp}}
  \caption{
    Wire planes in a cryostat and their disposition (from~\cite{SBNDocDBxxxx:ConnTest}).
    Note the orientation of the wires, correct and contrasting with the erroneous one in~\cite{SBNDocDB1020}.
    \label{fig:WirePlanesInTPC}
  }
\end{figure}





